\documentclass[11pt]{article}
\usepackage{multicol}
\usepackage{fullpage}
\usepackage{enumitem}
\usepackage{amsmath}
\usepackage{url}
\usepackage[compact]{titlesec}
\begin{document}
\title      {Intel: Multinational Semiconductor Manufacturer}
\author	    {Thomas Smith}
\date       {\today}
\maketitle
\begin{multicols}{2}
\section*{Introduction}%76
Intel is currently the world's largest supplier of semiconductor technology\cite{largest}, and has become so by successfully commercialising their early expertise with semiconductor manufacturing processes through effective management. Throughout their 43-year history so far, they have demonstrated this in a number of ways and this report will examine the techniques they have used. These range from intelligent marketing campaigns, good treatment of workforce and local communities, to (possibly most importantly) their continued innovation and expansion.

\subsection*{Early History}%205
Intel's first two decades put it firmly in a dominant position within the semiconductor industry. Founded in 1968 by two experts from an existing company, and an investor with \$2.5 million, the company's initial successes came from its proficiency at manufacturing semiconductor memory for the mainframe computers of the time. By 1971, the company was valued at \$6.8 million in it's Initial Public Offering (IPO), or \$23.50 per share. 
After developing the first commercially available microprocessor in 1971 and the first truly general-purpose microprocessor in 1974\cite{timeline}, Intel became part of the rapid growth of the Personal Computer (PC) industry during the 1980s. In 1975, the Intel 8080 chip had been used in one of the first PCs, and by 1981 IBM (the largest PC manufacturer at the time) had chosen to use Intel's 8088 chip in their IBM PC. In 1982, the company released the 80286, one of the first 16-bit microprocessors, and it was built into numerous PCs as the industry truly took off.
During this period the company entered the Fortune 500 list at \#486 in 1979, and by the time of its 20th anniversary celebrations was one of the largest semiconductor manufacturers in the world.

\subsection*{Marketing}%238 - 31 = 207
Part of Intel's considerable success at commercialising their semiconductor technology is down to their effective use of marketing: by making Intel a household name and ensuring a consistent positive association, they have achieved desirability even among users unfamiliar with computers.
In 1989 they launched the 'Red X' advertising campaign, their first targeted directly at consumers rather than manufacturers. This was unexpected creativity on Intel's part, but successfully changed purchaser's buying behaviour. By 1991, the 'Intel Inside' campaign and associated jingle\cite{mp3} had made Intel a household name, while competitors remained largely unknown. In 1993, Financial World listed Intel as the third most valuable brand worldwide.
In 1994, Intel's Pentium\footnote{Intel's decision to switch to memorable names for chips - rather than the obscure numbers used by their competitors (and previously themselves) - was part of their new consumer focus.} chips were the subject of a well publicised numerical flaw. After initial reticence, Intel offered to replace the chips at great cost to the company. However, the media coverage of the incident greatly increased the public's awareness of the Intel brand and, coinciding with the Intel Inside campaign, the response had a positive effect on the company's image.
Intel's marketing since has been focused on cementing their product's desirability in consumers' minds: their branding has been evolutionary rather than revolutionary\cite{wiki}, and even new product directions have been consistent with emphasising the focus on speed\cite{thunderbolt}.

\subsection*{Social Responsibility} %236
In line with their efforts to be viewed positively by both consumers and their own workforce, Intel undertakes a range of charitable and morale-boosting activities. This seems incongruous for a company based on manufacturing technology, but especially in light of modern cultural expectations of companies to be morally responsible it has proven to be a sound strategy.
The company aims to be a `great place to work'\cite{facts}, and has done so since the beginning. Even early events such as the emphasis on the 10th anniversary celebrations were designed to make employees feel valued, and this helped the company through the 1981 tech industry recession where employees voluntarily worked an extra 25\% each week unpaid. In 1984, the company was chosen as one of the 100 best to work for in America, and in 1987 when the company exited the recession it held 'Back in Black' celebrations to recognise employees' contributions.
Intel embarked on their programme of wider social responsibility with the creation of the Intel Foundation\cite{foundation} (a philanthropic organisation), on the year of their 20th anniversary, and has continued by training teachers and encouraging workers to volunteer in the communities where they live. More recently, it has focused on ensuring that both the company and its products are seen as 'Green' and environmentally-friendly. These efforts have ensured that Intel continues to receive the support of both its own workers and the wider public.

\subsection*{Innovation}%194
Aside from its innovations with customer perceptions, Intel has thrived on its technical prowess. In order to successfully commercialise its semiconducter technology, and retain its dominant position within the industry, the company has continuously innovated to ensure that its product are at the cutting edge. This has often involved considerable risk, such as the company's decision in the early 80's to eliminate the manufacture of computer memory (Intel's previous fort�) and focus on microprocessors. Innovation in this area has largely consisted of improvements and efficiencies in either the manufacturing processes\footnotemark\ or the chips themselves, but has also included discovering underutilised niches in the industry, leading to the Celeron processor for value PCs, the Centrino for laptops, and the Atom processor for netbooks. Later, this has involved identification and exploitation of related technologies, such as their expansion to Flash memory, wireless networking, and most recently peripheral interconnection\cite{thunderbolt}. Not all of Intel's investments have been successful (a solar energy startup was spun off and then failed in 2011\cite{solar}), but by continually risking investment in new technologies and processes Intel has ensured that their core semiconductor technologies have remained competitive and successful.
\footnotetext{Intel's vaunted `Copy Exactly' process has greatly improved speed and consistency\cite{copy}}

\section*{Conclusion}%122
Intel's effective management has manifested in a number of different manners. From the ultimately beneficial resolution to the Pentium flaw, to the emphasis on employee wellbeing, to the continued focus on product and process improvement and development, the company has demonstrated an ability to ascertain which actions will ensure continued success for its commercialised technologies.
They have found product opportunities both within their own core expertise and within related fields, but have also found many opportunities to ensure the success of these products by cultivating both a positive image of the company itself, and a loyal and motivated workforce. In conclusion, the successful commercialisation of Intel's semiconductor technologies has been due to its effective management of both existing situations and viable opportunities.


\begin{thebibliography}{9}

\bibitem{largest} 
	Deffree, S., 
	2011, 
	\emph{Intel remains `firmly in control' of top semiconductor supplier spot}. 
	EDN. \newline
	Online: \url{http://www.edn.com/article/print/518213-Intel_remains_firmly_in_control_of_top_semiconductor_supplier_spot_IC_Insights_reports.php}
\bibitem{timeline}
	\emph{Intel Museum - Corporate Timeline}. Intel.\newline Online: \url{http://www.intel.com/about/companyinfo/museum/archives/timeline.htm}
\bibitem{mp3}
	Werzowa, W., 1999, \emph{`Intel Inside' mnemonic jingle}. Intel. \newline Online: \url{http://www.uspto.gov/web/offices/ac/ahrpa/opa/kids/soundex/75332744.mp3}
\bibitem{wiki}
	\emph{Logo Comparison}. Wikipedia. \newline Online: \url{http://en.wikipedia.org/wiki/Intel#Logos}
	\newline
\bibitem{thunderbolt}
	\emph{Thunderbolt Technology}. Intel.\newline Online: \url{http://www.intel.com/content/www/us/en/io/thunderbolt/thunderbolt-technology-developer.html}
\bibitem{facts}
	\emph{Intel Company Information}. Intel. \newline Online: \url{http://www.intel.com/content/www/us/en/company-overview/company-facts.html}
\bibitem{foundation}
	\emph{Intel Foundation}. Intel. \newline Online: \url{http://www.intel.com/content/www/us/en/corporate-responsibility/intel-foundation.html}
\bibitem{copy}
	\emph{Intel Copy Exactly Process}. Intel.\newline Online: \url{http://www.intel.com/design/quality/mq_ce.htm}
\bibitem{solar}
	EETimes, 2011, \emph{Intel's solar spinoff files for bankruptcy}. EETimes. \newline Onine: \url{http://eetimes.com/electronics-news/4219234/Intel-s-solar-spinoff-files-for-bankruptcy}

\end{thebibliography}
\end{multicols}
\end{document}



Assignment
An essay of 1000 words to be completed by Monday 21stNovember for Mang3022/ Tuesday 22nd November for Mang3055 and handed into Management Reception by 3:30 p.m.
Essay title:
�Describe how an example of engineering technology has been successfully commercialised through effective management?�

Assignment
-Reference all sources thoroughly � not doing so is plagiarism
-Have introduction explaining what will be covered, and conclusion summarising again what was covered
-For each paragraph: introduce content, then explain points, then summarise at end of paragraph
-Relate and link perspectives of authors
-Include concepts and models from the course
-Relate to examples
-Demonstrate broader reading



The creative process (de Bono) � Related to Ben and Jerry�s
Generating knowledge and awareness -
Had a passion for food which led to knowledge of how to make quality mixes of ice cream
Incubation process - Looked for related business opportunities. Growth potential for quality ice cream market.
Generating ideas - Idea of �party parlour�
Evaluation and implementation - Distributed
the ice cream more broadly through effective relationships, persistent in the face of adversity.	

Finding opportunity: Where do opportunities come from?
> Business creativity looks for:-
� The unexpected: New directions that think beyond the current business models e.g. Facebook, Friends Reunited, Body Shop
� The incongruity: Identifying how parts of the current business models do not quite align and fit together e.g. robotics to improve production efficiency
� Underlying inadequacy: Identifying shortfalls relative to expectations within the current business models e.g. fast food chains
� Industry changes: Enhanced technology e.g. more powerful hardware creates opportunities for more sophisticated software
� Demographic changes: Changes in the population e.g. ageing population creates opportunities for health care products
� Cultural changes: Changes in attitude and habits e.g. binge drinking
� New knowledge: Engineering and science e.g. penicillin leading to anti- biotics.










1968 company founded. Rock contribs $10,000 and $2.5 mil convertible debentures
1971 IPO. $23.50 share, $6.8mil
1973 innovations
1974 Intel 8080. General purpose, cash registers and traffic lights
1975 8080 in personal computer
1977 innovations
1979 486 on the Fortune 500
1980 ethernet with DEC and Xerox
1981 recession. 125% worker time. IBM PC uses 8088
1982 80286 16-bit processor built into numerous pcs. industry takes off
1983 CMOS chips. $1bil in annual revenue
1984 one of 100 best companies to work for
1987 back in black after two year recession. employees celebrate
1988 intel Foundation. 20year anniversary celebrations
1989 red x campaign successfully changed consumer buying behaviour
1991 'Intel inside' a household name
1992 largest semiconductor supplier in the world, according to dataquest
1993 2 years since 'intel inside' Financial World says third most valuable brand worldwide
1993 pentium processor
1994 employee childcare. pentium flaw. intel powers 85% of desktop computers
1995 intel in the community
1998 celeron for value PC saw a niche, filled it
1999 Joins Dow Jones Industrial Average
2001 EUV lithography. Investment in processes
2003 training teachers. Centrino for laptops
2004 number 4 best 100 companies to work for. Third year on 100 best for working moms
2005 40th anniv. of Moore's Law. Apple. Realignment
2006 evolution of branding and marketing strategy. Intel makes possible in <look up>
2007 Children in need. 45nm Transistor Breakthrough
2008 40th anniversary. Atom processors for netbooks
2009 getting serious about going green
2011 Thunderbolt



intelligent marketing
sensible expansion and risks
treating workforce and community well
continued innovation 