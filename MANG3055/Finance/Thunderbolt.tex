\documentclass[11pt]{article}
\usepackage{multicol}
\usepackage{fullpage}
\usepackage{enumitem}
\usepackage{amsmath}
\usepackage[compact]{titlesec}
\begin{document}
\title      {Thunderbolt: High-Speed I/O Connectivity Interface}
\author	    {Thomas Smith}
\date       {\today}
\maketitle
\begin{multicols}{2}
\section*{Introduction}%119
This report will examine the new Thunderbolt interface hardware produced by Intel, and investigate its technical capabilities as compared to the value they represent. Thunderbolt is the collective name given to the system of internal microcontrollers, ports, protocols and active cables that make up the Thunderbolt Input/Output (I/O) interface, akin to USB or FireWire. It allows desktop and laptop computers to connect to a varied range of peripheral devices, including video capture devices, external storage solutions, displays, and a wide selection of legacy devices supported through backwards-compatible adapters. It also provides a number of valuable additional features, such as a 10W powered link, dual full-duplex 10Gbps channels, and the ability to daisy-chain multiple devices from a single Thunderbolt port.

\section*{Intel}%310
The Thunderbolt interface was developed by Intel, a company best known for their personal computer microprocessor manufacturing business. Founded in 1968, Intel has only recently begun to focus on anything other than the semiconductor products such as RAM and microprocessors which have been their forte for the previous 30-odd years. Launching from an early expertise with semiconductor manufacturing processes - creating and refining RAM chips - the company then invested in developing and improving the growing field of microprocessors. In the 1980s, Intel became the primary hardware supplier for the entire PC industry, and thereby profited hugely from IBM's introduction of the personal computer. 

The company continued to produce and refine microprocessors throughout the 1990s, and through various advertising efforts and a single high-profile design flaw Intel gained a substantial level of public awareness, with a couple of products becoming household names. This has allowed the company to build brand loyalty among average consumers, giving it an advantage over both its previous position - known only to engineers and people in the industry - and its competitors, who have less established reputations. Recent advertising campaigns and expansions to other technologies have begun to capitalise on the company's established engineering expertise and consumer perceptions; labelling themselves as 'the company with the fastest processors', in 2008 they expanded to produce a range of Solid State Drives (SSDs, fast persistent storage), and now in 2011 they have produced Thunderbolt (fast inter-device data transfer). While both products are departures from their traditional core business, they fit within the company's core competencies with semiconductor manufacture and microcontroller design.

Intel therefore has a lot of experience with the architecture of computing systems, and employs hundreds of engineers across a range of disciplines. Also, Thunderbolt itself was developed with technical collaboration from Apple, another major player in the computing industry with broad PC systems production experience.

\section*{Interface}%515
The major technical advance behind the introduction of Thunderbolt is the combination of two separate existing data protocols (PCI Express and DisplayPort) into a single common transport layer. PCI Express (PCIe) supports data transfer and device control, and is a widely used standard in the computing industry, while DisplayPort provides audio \& visual transmission in order to drive external display devices. Communications of either kind are packaged onto the trasport layer, routed and directed by the Thunderbolt controllers, and then unpackaged into their original format at the destination device. This means that the system can carry any form of data at high speeds by using existing support for PCIe or DisplayPort interfaces - without requiring any additional drivers or software modification - since the controller is connected directly to the host's PCIe bus and DisplayPort outputs and the connection itself is essentially transparent to the host system and peripherals.

The use of controllers in each Thunderbolt-capable device enables a number of additional features. In devices with multiple Thunderbolt ports, the controller can act as a bridge between them, allowing connected controllers to communicate without interrupting the activity of the intermediary device. This means that peripherals can be daisy-chained up to a total of 7 from a single port on the host computer, allowing whole ecosystems of peripheral devices to be connected to the host computer in a single action.

One of the main advantages that Thunderbolt offers however is the sheer speed of the data transfer rates that it can provide. Each link supports two channels transferring data concurrently without conflict, and each channel supports speeds of up to 10Gbps in both directions simultaneously. This high bandwidth capacity ensures that multiple devices can be connected (and active) downstream of the host without risking saturating the final step of the link to the host. The intelligent Quality of Service (QoS) provisioning done by the controllers can also ensure that consistent, smooth transmission of video data to displays or from cameras can continue even in the presence of spiking and unpredictable file transfers. 

Thunderbolt speeds also benefit from the use of active cables. Unlike traditional (passive) cables, which are nothing more than a particular electrical connection between ports on devices, active cables incorporate tuned circuitry into the cable ends, designed to minimise signal loss across the cable length. Since Thunderbolt connections provide 10 watts of power as standard, the cables can use a small amount of this power to greatly improve the quality of the signals they carry. By using chips tuned to the specific characteristics of individual cables in order to precisely amplify and filter the transported signals, the maximum possible data transmission rates can be achieved, while also allowing the cables to be longer, narrower and have tighter bend radii. The separation of cables' concerns into the cable itself also allows for future backwards-compatible upgrades to cable technology, for example replacing the current copper transmission medium with optical fibres paired with silicon photonics in the cable-end chips. This would greatly increase both the transmission speeds and cable lengths possible, as a tradeoff from the power-transmission functionality. 

\section*{Illustration}% 94
A professional photographer might have a high-end digital camera, a laptop, and an external HDD. Currently, to copy photos to the HDD requires inserting an SD card from the camera into the laptop, reading it through the internal USB host, and sending via FireWire to the HDD. This approach is bottlenecked both by the speed of the internal USB (480Mbps), and processor availibility for the redirection. With Thunderbolt connections throughout instead, the camera can be daisy-chained through the HDD, and requires a single interaction from the laptop to initiate a (considerably faster) 10Gbps transfer.  

\section*{Illative}%462
While features such as the power provision and the ability to daisy-chain devices will undoubtedly be attractive to consumers, the primary metric of an I/O interface is the speed with which it is able to transfer the input and output data. In this respect, Thunderbolt is currently one of the best systems on the market, outperforming comparable systems by several factors, and this is probably its main selling point. As computers and connections become increasingly faster, consumers will seek to eliminate bottlenecks in their workflows wherever possible. Given Thunderbolt's high throughput and direct connection to internal busses, external devices can begin to be accessed with the same ease and speed as internal ones, with the additional advantage of being hot-swappable. This can allow consumers to create systems which are essentially modular, increasing day-to-day flexibility and the ability to cope with changing system demands. The advent of this concept can be seen with previous interface systems - USB DVD drives, external FireWire HDDs - but the speed and extensibility of Thunderbolt will make it truly practical. 

Compared to existing and even upcoming systems, Thunderbolt's dual-channel throughput capacity of 10Gbps each is fairly clearly superior, coming in at twenty times the speed of USB 2.0, and twelve times the speed of Firewire 800. It also outperforms the upcoming USB 3.0 spec, with almost double the throughput capacity. Consumers will pay for this additional speed, especially in industries which typically deal with huge volumes of data such as audio or video production. The increased throughput speed will make a noticeable and measurable difference to transfer times, which can reduce workflow bottlenecks and thereby increase productivity. From the data provided by Intel, Thunderbolt is capable of transferring an entire full-length HD movie in 30 seconds, or an entire year's worth of mp3 files in just over 10 minutes\footnote{\ http://www.intel.com/content/www/us/en/io/thunderbolt/thunderbolt-technology-developer.html}. Potential purchasers can use the raw speed of the interface as a clear metric of the improvement the system will provide: as compared to their existing setup, and based on their typical weekly transfer volume, the relative speed increase will save them a certain number of hours per week. In this way they can assign a value to expected benefit from incorporating Thunderbolt into their computing environment.

The price-vs-speed-increase comparison will not necessarily actually be so clear to interpret however. Due to the required direct connection to internal system buses, Thunderbolt is not an upgrade that can be applied to all existing systems. To achieve the listed speeds, it is necessary that any peripherals used also support Thunderbolt themselves\footnote{\ While it is possible to connect older devices using adapters to other systems such as USB or Firewire, they will then be limited to the speed of the legacy interface (due to Thunderbolt's routing, devices connected upstream won't be affected).}. This means that consumers are less likely to upgrade their entire systems purely in order to obtain Thunderbolt connections; rather, Thunderbolt-compatibility will be an incentive to upgrade. Thunderbolt-capable devices have a clear technical advantage to inform purchasing decisions.


%  Not purtely that simple though: will require upgrading to devices that support this. Will probably occur as part of normal device turnover based on usable lifecycle. Hindered by the fact that device components are slightly more expensive than comparable passive cables, which complicates the [price/speed tradeoff]
% 
% [attractive features. main selling point speed.todays connected world, wait for file operation. direct connection to internal busses - external devices can be treated as internal. modiular computer. advent with other connection systems, [blazing outperformance. consumers will pay for additional speed, especially industries who are bottlenecked by file transfer. ths will make a measurable difference to transfer times (stats from intel). can use raw speed as a metric. ]


% Introduce the technology, stating who the company/individuals are, how long they have been in existence, what skills/capabilities they have.
% � Then explore the technology, the process. Do not shy away from complexity, your analysis should be aimed at the �intelligent layperson�.

% Provide a TECHNICAL measure with which to measure performance, one that can be used to provide a measure of resource use. So for example ton miles per engine hour is used in logistics to measure performance. We can then calculate � allocate � a cost per ton mile per engine hour.
% � Your conclusion should state how the financial measure relates to the non-financial, and any difficulties in interpreting performance.
% � I do not discuss marking schemes but broadly speaking you should spend about half of the report on describing the technology, about 20\% on the company/individuals and the remaining 30\% on the concluding discussion

\end{multicols}
\end{document}